\documentclass[11pt,letterpaper]{article}

% Load some basic packages that are useful to have
% and that should be part of any LaTeX installation.
%
% be able to include figures
\usepackage{graphicx}
% get nice colors
\usepackage{xcolor}

% change default font to Palatino (looks nicer!)
\usepackage[latin1]{inputenc}
\usepackage{mathpazo}
\usepackage[T1]{fontenc}
% load some useful math symbols/fonts
\usepackage{latexsym,amsfonts,amsmath,amssymb}

% comfort package to easily set margins
\usepackage[top=1in, bottom=1in, left=1in, right=1in]{geometry}

% control some spacings
%
% spacing after a paragraph
\setlength{\parskip}{.15cm}
% indentation at the top of a new paragraph
\setlength{\parindent}{0.0cm}


\begin{document}

\begin{center}
\Large
Ay190 -- Worksheet 3\\
Scott Barenfeld\\
Date: \today
\end{center}

I worked with Daniel DeFelippis
\section{Problem 1}
\subsection{Part a)}
To integrate $f(x)=\sin{x}$, I used the equations given in the notes 
for the midpoint rule, trapezoid rule, and Simpson's rule, 
respectively:

\begin{equation}
Q_i=(b_i-a_i)f\left(\frac{a_i+b_i}{2}\right)
\end{equation}

\begin{equation}
Q_i=\frac{1}{2}(b_i-a_i)[f(b_i)+f(a_i)]
\end{equation}

\begin{equation}
Q_i=\frac{b_i-a_i}{6}\left[f(a_i)+4f\left(\frac{a_i+b_i}{2}\right)+f(b_i)\right]
\end{equation}

The results are shown in Table \ref{tab:onea}.

\begin{table}[!h]
\centering
\caption{Results of Numerical Integration}
\begin{tabular}{c|c|c}
\hline
 & $h=\frac{\pi}{10}$ & $h=\frac{\pi}{100}$\\
\hline
Midpoint & 2.0082484079 & 2.0000822491\\
Trapezoidal & 1.9835235375 & 1.9998355039\\
Simpson's & 2.00000678444 & 2.00000000068\\
Analytic & 2 & 2\\
\hline
\end{tabular}
\label{tab:onea}
\end{table}

Table \ref{tab:onea2} shows the errors in these methods.

\begin{table}[!h]
\centering
\caption{Errors in Numerical Integration}
\begin{tabular}{c|c|c}
\hline
 & $h=\frac{\pi}{10}$ & $h=\frac{\pi}{100}$\\
\hline
Midpoint & $8.25\times10^{-3}$ & $8.22\times10^{-5}$\\
Trapezoidal & $1.65\times10^{-2}$ & $1.64\times10^{-4}$\\
Simpson's & $6.78\times10^{-6}$ & $6.76\times10^{-10}$\\
\hline
\end{tabular}
\label{tab:onea2}
\end{table}

The global error in the midpoint and trapezoidal rules goes as $h^2$, so 
the error should drop by a factor of 100 when $h$ decreases by a 
factor of 10.  The global error in Simpson's rule goes as $h^4$, 
so the error should drop by a factor of $10^4$.  Table \ref{tab:onea} 
shows that this indeed is the case.

\subsection{Part b)}
The results of the numerical integration of $f(x)=x\sin{x}$ are 
shown in Table \ref{tab:oneb}.  The errors are shown in Table 
\ref{tab:oneb2}.  As before, the decrease in error behaves as expected.  

\begin{table}[!h]
\centering
\caption{Results of Numerical Integration}
\begin{tabular}{c|c|c}
\hline
 & $h=\frac{\pi}{10}$ & $h=\frac{\pi}{100}$\\
\hline
Midpoint & 3.1545492224 & 3.1417218501\\
Trapezoidal & 3.1157114868 & 3.1413342637\\
Simpson's & 3.14160331057 & 3.14159265465\\
Analytic & $\pi$ & $\pi$\\
\hline
\end{tabular}
\label{tab:oneb}
\end{table}

\begin{table}[!h]
\centering
\caption{Errors in Numerical Integration}
\begin{tabular}{c|c|c}
\hline
 & $h=\frac{\pi}{10}$ & $h=\frac{\pi}{100}$\\
\hline
Midpoint & $1.29\times10^{-2}$ & $1.29\times10^{-4}$\\
Trapezoidal & $2.59\times10^{-2}$ & $2.58\times10^{-4}$\\
Simpson's & $1.07\times10^{-5}$ & $1.06\times10^{-9}$\\
\hline
\end{tabular}
\label{tab:oneb2}
\end{table}

\section{Problem 2}
\subsection{Part a)}
I used Gauss-Laguerre Quadrature to calculate 

\begin{equation}
n_e=\frac{8\pi(k_BT)^3}{(2\pi\hbar c)^3}\int_0^{\infty}\!\frac{x^2}{e^{x}+1}\,\mathrm{d}x.
\end{equation}

I rewrote the integral in the form 

\begin{equation}
Q=\int_0^{\infty}\!f(x)e^{-x}\,\mathrm{d}x
\end{equation}

where

\begin{equation}
f(x)=\frac{x^2e^{x}}{e^{x}+1}.
\end{equation}

I used Python's built-in Gauss-Laguerre roots and weights.  My results 
are shown in Table \ref{tab:two}.

\begin{table}[!h]
\centering
\caption{Gauss-Laguerre Integration}
\begin{tabular}{c|c|c}
\hline
$n$ & $Q$ & $n_e (cm^{-3})$\\
\hline
5 & 1.802027 & $1.896\times10^{35}$\\
10 & 1.803095 & $1.897\times10^{35}$\\
20 & 1.803085 & $1.897\times10^{35}$\\
\hline
\end{tabular}
\label{tab:two}
\end{table}

To check for converge, I used the fact that Gauss-Laguerre quadrature 
has an error term of 
\begin{equation}
\mathcal{O}(n)\propto\frac{(n!)^2}{(2n)!},
\end{equation}
which I found on Wolfram Mathworld.  Using self-convergence
\begin{equation}
\frac{|Q(20)-Q(10)|}{|Q(10)-Q(5)|}=\frac{\mathcal{O}(20)-\mathcal{O}(10)}{\mathcal{O}(10)-\mathcal{O}(5)}
\end{equation}
The left hand side evaluates to $9.14\times 10^{-3}$, while the right hand side evaulates to $1.37\times 10^{-3}$.
These are a little off, but the error terms also have a factor of the $2n^{th}$ derivative of $f$ which I did 
not include.  The $n=20$ value of 1.803085 matches what I found for the integral using Mathematica.

\subsection{Part b)}
I first rewrote the integral over the $i^{th}$ bin:
\begin{equation}\label{eq:int}
\int_{5i}^{5(i+1)}\!\frac{x^2}{e^{x}+1}\,\mathrm{d}x
\end{equation}
in a form I could use Gauss-Legendre Quadrature on.  I made the substitution
\begin{equation}
u=\frac{x-\frac{5i+5(i+1)}{2}}{5/2}=\frac{2x-10i-5}{5}
\end{equation}
implying
\begin{equation}\label{eq:x}
x=\frac{5u+10i+5}{2}
\end{equation}
\begin{equation}
\mathrm{d}x=\frac{5}{2}\mathrm{d}u.
\end{equation}
This substitution changes Eq. \ref{eq:int} to
\begin{equation}
\frac{5}{2}\int_{-1}^{1}\!\frac{x^2}{e^{x}+1}\,\mathrm{d}u
\end{equation}
with the expression for $x$ in Eq. \ref{eq:x}.  I then implemented 
Gauss-Legendre Quadrature, again using Python's built-in weights and roots.  
I then calculated 
\begin{equation}
\left[\frac{\mathrm{d}n_e}{\mathrm{d}E}\right]_i=\frac{[n_e]_i}{\Delta E}
\end{equation}
for bins of size $\Delta E=5$ MeV from $0$ to $150$ MeV.  The results are displayed 
in Table \ref{tab:twob}.

\begin{table}[!h]
\centering
\caption{Spectral Distribution}
\begin{tabular}{c|c|c|c|c|c}
\hline
E Range (MeV) & $\frac{\mathrm{d}n_e}{\mathrm{d}E}$ (MeV/cm$^3$) & E Range (MeV) & $\frac{\mathrm{d}n_e}{\mathrm{d}E}$ (MeV/cm$^3$) & 
E Range (MeV) & $\frac{\mathrm{d}n_e}{\mathrm{d}E}$ (MeV/cm$^3$)\\
\hline
0-5 & $3.27\times10^{34}$ & 50-55 & $1.04\times10^{16}$ & 100-105 & $7.93\times10^{-6}$\\
5-10 & $5.12\times10^{33}$ & 55-60 & $8.51\times10^{13}$ & 105-110 & $5.88\times10^{-8}$\\
10-15 & $1.15\times10^{32}$ & 60-65 & $6.80\times10^{11}$ & 110-115 & $4.35\times10^{-10}$\\
15-20 & $1.64\times10^{30}$ & 65-70 & $5.37\times10^{9}$ & 115-120 & $3.20\times10^{-12}$\\
20-25 & $1.90\times10^{28}$ & 70-75 & $4.19\times10^{7}$ & 120-125 & $2.35\times10^{-14}$\\
25-30 & $1.96\times10^{26}$ & 75-80 & $3.23\times10^{5}$ & 125-130 & $1.71\times10^{-16}$\\
30-35 & $1.88\times10^{24}$ & 80-85 & $2.47\times10^{3}$ & 130-135 & $1.25\times10^{-18}$\\
35-40 & $1.71\times10^{22}$ & 85-90 & $1.88\times10^{1}$ & 135-140 & $9.06\times10^{-21}$\\
40-45 & $1.49\times10^{20}$ & 90-95 & $1.42\times10^{-1}$ & 140-145 & $6.56\times10^{-23}$\\
45-50 & $1.26\times10^{18}$ & 95-100 & $1.06\times10^{-3}$ & 145-150 & $4.74\times10^{-25}$\\
\hline
\end{tabular}
\label{tab:twob}
\end{table}

To check that my code worked, I recalculated the total $n_e$ using
\begin{equation}
n_e=\displaystyle\sum\limits_{i=0}^{\infty} \left[\frac{\mathrm{d}n_e}{\mathrm{d}E}\right]_i\times\Delta E.
\end{equation}
I found $n_e=1.897\times10^{35}$, matching what I found in Part a).


\end{document}
Anything that comes after \end{document} is completely
ignored by LaTeX.
