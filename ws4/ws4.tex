\documentclass[11pt,letterpaper]{article}

% Load some basic packages that are useful to have
% and that should be part of any LaTeX installation.
%
% be able to include figures
\usepackage{graphicx}
% get nice colors
\usepackage{xcolor}

% change default font to Palatino (looks nicer!)
\usepackage[latin1]{inputenc}
\usepackage{mathpazo}
\usepackage[T1]{fontenc}
% load some useful math symbols/fonts
\usepackage{latexsym,amsfonts,amsmath,amssymb}

% comfort package to easily set margins
\usepackage[top=1in, bottom=1in, left=1in, right=1in]{geometry}

% control some spacings
%
% spacing after a paragraph
\setlength{\parskip}{.15cm}
% indentation at the top of a new paragraph
\setlength{\parindent}{0.0cm}


\begin{document}

\begin{center}
\Large
Ay190 -- Worksheet 4\\
Scott Barenfeld\\
Date: \today
\end{center}

I worked with Daniel DeFelippis.

\section{Problem 1}
\subsection{Part a}
To find the \bf{what's the term?}, $E$, I used 
\begin{eqaution}
E-\omega t-e\Sin{E}
\end{equation}
where $\omega=2\pi/T$ is the angular frequency, $T$ is the period, 
$e$ is the eccentricity, and t is the day of the orbit.  I used 
Newton's method.  For the Earth, $\omega\approx 0.01$, so for $t$ of 
around 100, so $\omega t\approx 1$.  So for small $e$, $E=1$ should be 
about right, so I used $E=1$ as my initial guess.  To find the $x$ and 
$y$ positions of the Earth, I use the equations
\begin{equation}
\end{eqaution}

\begin{equation}
\end{eqaution}
My results are summarized in Table \ref{tab:one}.  For all three values 
of $t$, the solution converged in 4 steps.  To make sure my 
code still works for bad initial guesses, I tried an initial guess of 
100.  My solution (Table \ref{Tab:hun}) converged to the same values, 
this time in 5 steps.

\begin{table}
\centering
\caption{Newton's Method for Initial Guess of 1}
\begin{tabular}[c|c|c|c|]
\hline
t (days) & E & x (AU) & y (AU)\\
\hline
\hline
91 & 1.582 &
182 & 3.131 &
273 & 4.679 &
\hline
\end{tabular}
\label{tab:one}
\end{table}

\begin{table}
\centering
\caption{Newton's Method for Initial Guess of 100}
\begin{tabular}[c|c|c|c|]
\hline
t (days) & E & x (AU) & y (AU)\\
\hline
\hline
91 & 1.582 &
182 & 3.131 &
273 & 4.679 &
\hline
\end{tabular}
\label{tab:hun}
\end{table}

\end{document}
Anything that comes after \end{document} is completely
ignored by LaTeX.
